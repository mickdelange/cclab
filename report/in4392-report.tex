\documentclass{acm_proc_article-sp}
\makeatletter
\def\@copyrightspace{\relax}
\makeatother

\begin{document}

% 1. Report title, authors, and support cast (Lab assistant and course instructors). For each person, give name and contact information (email).
\title{}
\subtitle{}

\numberofauthors{2}
\author{
\alignauthor
R.M. de Lange\\
		\affaddr{1534068}\\
		\email{\{r.m.delange,}
\alignauthor
M. Voinea\\
		\affaddr{4317602}\\
		\email{m.voinea\}@student.tudelft.nl}
% \alignauthor
% D.H.J. Epema\\
% 		\email{d.h.j.epema@tudelft.nl}
% \alignauthor
% A. Iosup\\
% 		\email{a.iosup@tudelft.nl}
% \alignauthor
% B.I. Ghit\\
% 		\email{b.i.ghit@tudelft.nl}
}

\maketitle

% 2. Abstract: a description of the problem, system description, analysis overview, and one main result. Size: one paragraph with at most 150 words.
\begin{abstract}
\end{abstract}

% 3. Introduction (recommended size, including points 1 and 2: 1 page): describe the
% problem, the existing systems and/or tools (related work), the system you are about to
% implement, and the structure of the remainder of the article; use one short paragraph
% for each.
\section{Introduction}

% 4. Background on Application (recommended size: 0.5 pages):
% describe the application (1 paragraph) and its requirements (1-3 paragraphs, summarized in a table if needed).
\section{Background}

% 5. System Design (recommended size: 1.5 pages)
% a. Resource Management Architecture: describe the design of your system,
% including the inter-operation of the provisioning, allocation, reliability, and
% monitoring components (which correspond to the homonym features required
% by the WantCloud CTO).
% b. System Policies: describe the policies your system uses and supports. The latter
% may remain not implemented throughout your coursework, as long as you can
% explain how they can be supported in the future.
% c. (Optional, for bonus points, see Section F) Additional System Features:
% describe each additional feature of your system, one sub-section per feature.
\section{System Design}

% 6. Experimental Results (recommended size: 1.5 pages)
% a. Experimental setup: describe the working environments (DAS, Amazon EC2,
% etc.), the general workload and monitoring tools and libraries, other tools and
% libraries you have used to implement and deploy your system, other tools and
% libraries used to conduct your experiments.
% b. Experiments: describe the experiments you have conducted to analyze each
% system feature, then analyze them; use one sub-section per experiment. For
% each experiment, describe the workload, present the operation of the system,
% and analyze the results. In the analysis, report:
% i. Charged-time = time that would have been charged using the
% Amazon EC2 timing approach (1-hour increments)
% ii. Charged-cost = cost that would have been charged using the
% Amazon EC2 charging approach, assuming 10 Euro-cents/charged hour
% iii. Service metrics of the experiment, such as runtime and response time of
% the service, etc.
% iv. (optional) Usage metrics of the experiment, such as per-VM and overall
% system usage and activity.
\section{Evaluation}

% 7. Discussion (recommended size: 1 page): summarize the main findings of your work
% and discuss the tradeoffs inherent in the design of cloud-computing-based applications.
% Should the WantCloud CTO use IaaS-based clouds? Among others, use extrapolation
% on the results, as reported in Section 6.b of the report, to discuss the charged time and
% charged cost reported in section for 100,000/1,000,000/10,000,000 users and for 1
% day/1 month/1 year.
\section{Discussion}

% 8. Conclusion
\section{Conclusion}

\bibliography{references}{}
\bibliographystyle{plain}

% 9. Appendix A: Time sheets (see Section E)
\appendix

\end{document}